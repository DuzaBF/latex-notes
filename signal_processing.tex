\documentclass{article}
\usepackage{graphicx} % Required for inserting images
\usepackage{amsmath} 
\usepackage{pdflscape}
\usepackage{csquotes}
\usepackage{amsfonts}
\usepackage{tikz}
\usetikzlibrary{calc}
\usepgflibrary{arrows.meta}
\usetikzlibrary{tikzmark}


\title{Signal Processing}
\author{Airat Galiamov}
\date{November 2024}

\begin{document}

\maketitle

\section{Fourier transform}
\subsection{Definitions}
If $f(t)$ is a function of time, then $F(\omega)$ is its Fourier transform $f(t) \xrightarrow {\mathcal{F}} F(\omega)$ and spectral density if:
\begin{equation*}
    F(\omega) = \mathcal{F}[f(t)] = \int \limits^{\infty}_{-\infty} f(t) e^{-i\omega t}dt
\end{equation*}
Inverse transform $F(\omega) \xrightarrow{\mathcal{F}^{-1}} f(t)$:
\begin{equation*}
    f(t) = \mathcal{F}^{-1}[F(\omega)] = \int \limits^{\infty}_{-\infty} F(\omega) e^{i\omega t}d\omega
\end{equation*}

\subsection{Properties}

\paragraph{Linearity} Follows from the linearity of the integration:
\begin{equation*}
    a f(t) + b g(t) \xrightarrow{\mathcal{F}} a F(\omega) + b G(\omega)
\end{equation*}
\paragraph{Scaling} Wider in time - narrower in frequency. Follows from the change of variables:
\begin{align*}
    &f(at) \xrightarrow{\mathcal{F}}  \frac{1}{\left| a \right|} F(\frac{\omega}{a})  \\
    &\int \limits^{\infty}_{-\infty} f(at) e^{-i\omega t}dt = \\
    & = \left| u = at; \, dt = \frac{1}{a}du; \, \text{change of limits $\rightarrow$ absolute value of $a$ }\right| =\\
    & = \frac{1}{\left| a \right|} \int \limits^{\infty}_{-\infty} f(u) e^{-i \frac{\omega}{a} u }du =  \\
    & =\frac{1}{\left| a \right|} F(\frac{\omega}{a})
\end{align*}

\paragraph{Time shift} Shift in time - phase shift in frequency. Follows from the change of variables:
\begin{align*}
    &f(t - t_0) \xrightarrow{\mathcal{F}}  e^{-i\omega t_0} F({\omega})  \\
    &\int \limits^{\infty}_{-\infty} f(t - t_0) e^{-i\omega t}dt = \left| u=t - t_0 \right| = \\
    &=\int \limits^{\infty}_{-\infty} f(u) e^{-i\omega u - i \omega t_0}du = \\ 
    &=e^{- i \omega t_0}\int \limits^{\infty}_{-\infty} f(u) e^{-i\omega u }du =  e^{-i\omega t_0} F({\omega})
\end{align*}

\paragraph{Differentiation} Derivative in time - multiplication by $i \omega$ in frequency. 
Useful property for solving differential equations as it transform differential equation into algebraic.
 Follows from the inverse transform:
\begin{align*}
    & \frac{d f(t)}{dt} \xrightarrow{\mathcal{F}} i\omega F({\omega}) \\
    &\frac{d f(t)}{dt} = \frac{d }{dt} \int \limits^{\infty}_{-\infty} F(\omega) e^{i\omega t}d \omega = \\
    &=\int \limits^{\infty}_{-\infty}  F(\omega) \frac{d e^{i\omega t}}{dt} d \omega = \\
    &= \int \limits^{\infty}_{-\infty} \left[ i\omega F(\omega) \right] e^{i\omega t} d \omega \\
\end{align*}

\paragraph{Convolution} Convolution in time - multiplication in frequency. 
Useful in working with Linear Time Invariant systems, discretization, optics etc.
 Effect of the LTI systems is described as a convolution of the input signal with its impulse response.
 Follows from the changing the order of integration:
\begin{align*}
    % TODO proof
    &f(t) * g(t) \xrightarrow{\mathcal{F}} F(\omega) G(\omega) \\
    &\int \limits^{\infty}_{-\infty} \int \limits^{\infty}_{-\infty} f(\tau ) g(t - \tau) d \tau e^{-i\omega t}dt = 
\end{align*}

\paragraph{Symmetry} Follows from the definitions of the transform and it's inverse:
\begin{align*}
    &f(t) \xrightarrow{\mathcal{F}} F(\omega) \\
    &F(t) \xrightarrow{\mathcal{F}} f(-\omega)
    % TODO proof
\end{align*}

\paragraph{Multiplication} Multiplication in time - convolution in frequency. Follows from the symmetry:
\begin{align*}
    f(t)g(t) \xrightarrow{\mathcal{F}} F(\omega) * G(\omega)
    % TODO proof
\end{align*}

\paragraph{Integration} Reverse of the differentiation. 
The term with Dirac delta function takes care of the constant of integration with a constant in time part of the function.
\begin{align*}
    \int \limits^{t}_{-\infty}f(\tau)d \tau \xrightarrow{\mathcal{F}} \frac{F(\omega)}{i \omega} + \frac{1}{2}F(0)\delta(\omega)
    % TODO proof
\end{align*}

\paragraph{Modulation} Multiplication with a sine wave in time - shift in frequency. Basis of AM modulation.
Follows from the time-shift and symmetry:
\begin{align*}
    e^{i \omega_0 t}f(t) \xrightarrow{\mathcal{F}} F(\omega - \omega_0)
    % TODO proof
\end{align*}

\paragraph{Conjugate signal}
\begin{align*}
   f^*(t) \xrightarrow{\mathcal{F}} F^{*}(-\omega)
    % TODO proof
\end{align*}

\subsection{Important functions}

\paragraph{Dirac delta function}

Dirac delta function $\delta(t)$ is defined as:
\begin{align*}
    &\delta(t) = \begin{cases}
        0, \, \text{if} \, t=0 \\
        \infty, \, \text{overwise}
    \end{cases} \\
    &\text{and} \\
    &\int \limits^{\infty}_{-\infty} \delta(t) dt = 1
\end{align*}
So it has infinite value at $0$ but its "energy" is finite

Important property of the Dirac delta function - it samples the functions it is getting integrated with:
\begin{equation*}
    \int \limits^{\infty}_{-\infty} f(t) \delta(t-t_0) dt = f(t_0)
\end{equation*}

Fourier transform of the Dirac delta function:
\begin{align*}
    \delta(t-t_0) \xrightarrow{\mathcal{F}} = e^{-i \omega t_0} \\
    \int \limits^{\infty}_{-\infty} \delta(t-t_0) e^{-i \omega t} dt = e^{-i \omega t_0}
\end{align*}

\newcommand{\DrawAxes}[7]{
    \coordinate (x_dir) at (1,0);
    \coordinate (y_dir) at (0,1);
    \coordinate (center) at #1;
    \def\mx{#2};
    \def\px{#3};
    \def\my{#4};
    \def\py{#5};
    \draw[->, thick] ($ (center) - \mx *(x_dir) $) -- ($ (center) + \px *(x_dir) $) node [below right] {#6};
    \draw[->, thick] ($ (center) - \my *(y_dir) $) -- ($ (center) + \py *(y_dir) $) node [above right] {#7};
}

\newcommand{\DrawDiracDelta}[2] {
    \coordinate (center) at #1;
    \coordinate (dir) at #2;
    \coordinate (end) at ($ (center) + (dir) $);
    \coordinate (pos) at (center-|end);
    \draw[->, thick] (pos) -| (end);
}

\begin{tikzpicture}
    % Draw grid
    % \draw[step=1,gray,very thin] (\xone,\yone) grid (\xtwo,\ytwo);
    % Draw axes

    \coordinate (t_center) at (0,0);
    \coordinate (f_center) at (8,0);

    \DrawAxes{(t_center)}{1}{3}{0.5}{2}{$t$}{$f$};
    \DrawAxes{(f_center)}{2}{2}{0.5}{2}{$\omega$}{$|F|$};

    \draw [->] (4,1) -- (5,1) node[midway,above] {$\mathcal{F}$};

    \DrawDiracDelta{(t_center)}{(1,2)};
    \node at ($ (t_center) + (1,2) $) [above right] {$\delta(t-t_0)$};
    \node at (1,0) [below] {$t_0$};

    \draw ($ (f_center) + (-2,1) $) -- ($ (f_center) + (2,1) $);
    \node at ($ (f_center) + (0,1) $) [above right] {$1$};

\end{tikzpicture}


Inverse Fourier transform of Dirac delta function on frequency:
\begin{align*}
    \delta(\omega-\omega_0) \xrightarrow{\mathcal{F}^{-1}} = e^{i \omega_0 t}\\
    \int \limits^{\infty}_{-\infty} \delta(\omega-\omega_0) e^{i \omega t} d\omega = e^{i \omega_0 t}
\end{align*}

\paragraph{sin and cos}
Using Euler's formula $sin(\omega_0 t)$ and $cos(\omega_0 t)$ can be represented as a sum of complex exponential functions:
\begin{align*}
    &e^{ix} = cos(x) + i sin(x)\\
    &sin(\omega_0 t) = \frac{e^{i \omega_0 t} - e^{-i \omega_0 t}}{2i} \\
    &cos(\omega_0 t) = \frac{e^{i \omega_0 t} + e^{-i \omega_0 t}}{2} \\
\end{align*}
From these and properties of Dirac delta function the Fourier transforms of $sin(\omega_0 t)$ and $cos(\omega_0 t)$ are:
\begin{align*}
    &sin(\omega_0 t) \xrightarrow{\mathcal{F}} \frac{1}{2i} \left( \delta(\omega - \omega_0) - \delta(\omega + \omega_0) \right) \\
    &cos(\omega_0 t) \xrightarrow{\mathcal{F}} \frac{1}{2}  \left( \delta(\omega - \omega_0) + \delta(\omega + \omega_0) \right)\\
\end{align*}

\begin{tikzpicture}
    % Draw grid
    % \draw[step=1,gray,very thin] (\xone,\yone) grid (\xtwo,\ytwo);
    % Draw axes

    \coordinate (t_center) at (0,0);
    \coordinate (f_center) at (8,0);

    \DrawAxes{(t_center)}{1}{3}{1.5}{1.5}{$t$}{$f$};
    \DrawAxes{(f_center)}{2}{2}{1.5}{1.5}{$\omega$}{$\Im(F)$};

    \draw [domain=-1:3, samples=100]  plot (\x,{sin( 3 * \x r)});

    \draw [->] (4,0) -- (5,0) node[midway,above] {$\mathcal{F}$};

    \DrawDiracDelta{(f_center)}{(1,-1.5)};
    \DrawDiracDelta{(f_center)}{(-1,1.5)};
    \node at ($ (f_center) + (1,-1.5) $) [below right] {$-\frac{1}{2}\delta(\omega-\omega_0)$};
    \node at ($ (f_center) + (-1,1.5) $) [above left] {$\frac{1}{2}\delta(\omega+\omega_0)$};
    \node at ($ (f_center) + (1,0) $) [above] {$\omega_0$};
    \node at ($ (f_center) + (-1,0) $) [below] {$-\omega_0$};
\end{tikzpicture}


\begin{tikzpicture}
    % Draw grid
    % \draw[step=1,gray,very thin] (\xone,\yone) grid (\xtwo,\ytwo);
    % Draw axes

    \coordinate (t_center) at (0,0);
    \coordinate (f_center) at (8,0);

    \DrawAxes{(t_center)}{1}{3}{1.5}{1.5}{$t$}{$f$};
    \DrawAxes{(f_center)}{2}{2}{1.5}{1.5}{$\omega$}{$\Re(F)$};

    \draw [domain=-1:3, samples=100]  plot (\x,{cos( 3 * \x r)});

    \draw [->] (4,0) -- (5,0) node[midway,above] {$\mathcal{F}$};

    \DrawDiracDelta{(f_center)}{(1,1.5)};
    \DrawDiracDelta{(f_center)}{(-1,1.5)};
    \node at ($ (f_center) + (1,1.5) $) [above right] {$\frac{1}{2}\delta(\omega-\omega_0)$};
    \node at ($ (f_center) + (-1,1.5) $) [above left] {$\frac{1}{2}\delta(\omega+\omega_0)$};
    \node at ($ (f_center) + (1,0) $) [below] {$\omega_0$};
    \node at ($ (f_center) + (-1,0) $) [below] {$-\omega_0$};
\end{tikzpicture}

\paragraph{Heaviside step function}

Heaviside step function is defined as:
\begin{equation*}
    H(t) = \begin{cases}
        0, \, \text{if} \, t<0 \\
        1, \, \text{overwise}
    \end{cases} 
\end{equation*}
Also step function is an integral (antiderivative) of the Dirac delta function:
\begin{equation*}
    H(t) = \int \limits^{t}_{-\infty} \delta (\tau) d\tau
\end{equation*}
Fourier transform of the step function follows from its link 
to the Dirac delta function and integration property of the transform:
\begin{equation*}
    H(t)  \xrightarrow{\mathcal{F}} \frac{1}{i\omega } + \frac{1}{2} \delta (\omega)
\end{equation*}

\paragraph{rect} Rectangular function is defined as:
\begin{equation*}
    rect(t) = \begin{cases}
        0, \, \text{if} \, |t| > \frac{1}{2} \\
        1, \, \text{if} \, |t| \le \frac{1}{2} \\
    \end{cases} 
\end{equation*}
Rectangular function is also a combination of Heaviside step functions:
\begin{equation*}
    rect(t) = H(t+\frac{1}{2}) - H(t - \frac{1}{2})
\end{equation*}
Fourier transform of the rectangular function is a $sinc$ functions:
\begin{align*}
    rect(t) &\xrightarrow{\mathcal{F}} \frac{sin(\frac{\omega}{2})}{\frac{\omega}{2}} = sinc(\frac{\omega}{2}) \\
    \mathcal{F}[rect(t)] &=  \int \limits^{\infty}_{-\infty} rect(t) e^{-i \omega t}dt = \int \limits^{\frac{1}{2}}_{-\frac{1}{2}} 1 e^{-i \omega t}dt = \\
    &= \frac{1}{- i \omega}e^{-i \omega t} \Big|^{\frac{1}{2}}_{-\frac{1}{2}} = 
    \frac{e^{i\frac{\omega}{2}} - e^{-i\frac{\omega}{2}}}{2 i \frac{\omega}{2}} = \frac{\sin(\frac{\omega}{2})}{ \frac{\omega}{2}} = sinc(\frac{\omega}{2}) \\
    &\text{also} \,\\
    \frac{1}{2} rect(\frac{t}{2}) &\xrightarrow{\mathcal{F}} sinc(\omega)
    % \mathcal{F}(rect(t)) &= \mathcal{F}(H(t+\frac{1}{2}) - H(t - \frac{1}{2})) = \\
    % &=\frac{e^{i\omega \frac{1}{2}}}{i\omega } + \frac{e^{i\omega \frac{1}{2}}}{2} \delta (\omega) -
    % \frac{e^{-i\omega \frac{1}{2}}}{ i\omega } - \frac{e^{-i\omega \frac{1}{2}}}{2} \delta (\omega) = \\
    % &= \frac{e^{i\frac{\omega}{2}} - e^{-i\frac{\omega}{2}}}{2 i \frac{\omega}{2}} + \frac{e^{i\frac{\omega}{2}} - e^{-i\frac{\omega}{2}}}{2} \delta (\omega) = \\
    % &= \frac{\sin(\frac{\omega}{2})}{ \frac{\omega}{2}} + \frac{e^{i\frac{\omega}{2}} - e^{-i\frac{\omega}{2}}}{2} \delta (\omega)
\end{align*}

\paragraph{sinc} sinc function is defined as:
\begin{equation*}
    sinc(t) = \begin{cases}
        \frac{\sin(t)}{t}, \, \text{if} \, t \ne 0 \\
        1, \, \text{if} \, t = 0\
    \end{cases} 
\end{equation*}
Fourier transform of the sinc function os a rectangular function, which follows from the symmetry of the transform:
\begin{align*}
    sinc(t) &\xrightarrow{\mathcal{F}} \frac{1}{2} rect(\frac{\omega}{2}) 
\end{align*}

\paragraph{Dirac comb} Dirac comb is a function with periodic delta impulses. 
Multiplication of some function with the train performs a discrete sampling of the function.
\begin{align*}
    \delta_{T}(t) = \sum \limits^{\infty}_{n = -\infty} \delta(t - nT)
\end{align*}
Fourier transform of the Dirac comb is a Dirac comb:
\begin{align*}
    \delta_{T}(t) &\xrightarrow{\mathcal{F}} \frac{2\pi}{T} \delta_{\frac{2 \pi}{T}}(\omega) \\
    % TODO proof
\end{align*}

\section{Nyquist-Shannon-Kotelnikov theorem}

Paraphrased from \cite{nyquist_certain_1928,shannon_communication_1949,kotelnikov_transmission_2006}:
\begin{displayquote}
Any function $f(t)$ consisting of frequencies from $0$ to $f_1$ (band-limited) can be 
reconstructed with arbitrary accuracy from discrete samples at intervals of no higher than $\frac{1}{2f_1}$.
\end{displayquote}

Aliasing is an effect in discrete signals when frequencies of the initial continuous signal higher than twice the sampling frequency
get shifted to the lower frequency after sampling and change the signal.

Let $f(t)$ be band limited continuous signal with its spectra $F(\omega)$ and the highest frequency in its spectra $f_1$. 
Time signal $f(t)$ is purely real, meaning frequency spectra $F(\omega)$ is even.



Let $s(t)$ be the $f(t)$ signal sampled with a period $T$, sampling frequency $f_s = \frac{1}{T}$
 and angular sampling frequency $\omega_s = \frac{2 \pi }{T}$.
 Then the $s(t)$ is the multiplication of the $f(t)$ and the Dirac comb:
 \begin{equation*}
    s(t) = f(t) \delta_{T}(t)
 \end{equation*}
 Multiplication in time domain is a convolution in frequency domain.
 Convolving with a Dirac delta impulse shifts the function to the shift if teh Dirac impulse.
 \begin{align*}
    \mathcal{F}[s(t)] &= \mathcal{F}[f(t) \delta_{T}(t)] = F(\omega) * \omega_s \delta_{\omega_s}(\omega) = \\
    &=\sum \limits^{\infty}_{n = -\infty} \omega_s F(\omega - n \omega_s)
 \end{align*}

 If $f_1$ is greater than $f_s$ then adjacent copies of spectra will overlap and change and initial information about 
 the spectra will be lost.

 Continuous signal - continuous spectra.
 Discrete signal - periodic spectra.
 Reconstruction of the signal from it samples - cutting out one period of the spectra.
 Cutting out one spectra is multiplying spectra with rectangular function $rect(\frac{\omega}{\omega_s})$ with a width of $2\omega_s$.
 Multiplying with $rect$ in frequency domain - convolving with $sinc$ in time domain.
 Sampled signal in time domain is a train of weighted delta impulses. Convolving with $sinc$ makes every impulse into shifted and scaled $sinc$.
 Sum of these $sinc$'s will reconstruct continuous signal perfectly.


\section{Signal energy}
% TODO:
Factor of $\frac{1}{2 \pi}$ is required for the normalization and holding of Plancherel's theorem for energy relation.


\section{Definitions}
Complex numbers:
\begin{align*}
    &z = a + ib = r e^{i\phi} \\
    &z^* = a - ib \\
    &\left| z \right| ^2 = z z^* = a^2 + b^2
\end{align*}
Inner product of complex functions:
\begin{equation*}
    \left< f(t),g(t) \right> = \int \limits^{\infty}_{-\infty} f(t) g^{*}(t) dx
\end{equation*}
Convolution:
\begin{equation*}
f(t) * g(t) =  \int \limits^{\infty}_{-\infty} f(\tau ) g(t - \tau) d \tau
\end{equation*}
Dirac delta function:
\begin{equation*}
    \delta(t) =  \begin{cases}
        \infty, \, \text{if $t = 0$};\\
        0, \, \text{otherwise}
    \end{cases}
\end{equation*}

\bibliographystyle{IEEEtran}
\bibliography{signal_processing}

\end{document}
