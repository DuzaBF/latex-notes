\documentclass{article}
\usepackage{graphicx} % Required for inserting images
\usepackage{amsmath} 
\usepackage{pdflscape}
\usepackage{csquotes}
\usepackage{amsfonts}
\usepackage{tikz}
\usetikzlibrary{calc}
\usepgflibrary{arrows.meta}
\usetikzlibrary{tikzmark}


\title{Signal Processing}
\author{Airat Galiamov}
\date{November 2024}

\begin{document}

\maketitle

\section{Fourier transform}
\subsection{Definitions}
If $f(t)$ is a function of time, then $F(\omega)$ is its Fourier transform $f(t) \xrightarrow {\mathcal{F}} F(\omega)$ and spectral density if:
\begin{equation*}
    F(\omega) = \mathcal{F}(f(t)) = \int \limits^{\infty}_{-\infty} f(t) e^{-i\omega t}dt
\end{equation*}
Inverse transform $F(\omega) \xrightarrow{\mathcal{F}^{-1}} f(t)$:
\begin{equation*}
    f(t) = \mathcal{F}^{-1}(F(\omega)) = \int \limits^{\infty}_{-\infty} F(\omega) e^{i\omega t}d\omega
\end{equation*}

\subsection{Properties}

\paragraph{Linearity} Follows from the linearity of the integration:
\begin{equation*}
    a f(t) + b g(t) \xrightarrow{\mathcal{F}} a F(\omega) + b G(\omega)
\end{equation*}
\paragraph{Scaling} Wider in time - narrower in frequency. Follows from the change of variables:
\begin{align*}
    &f(at) \xrightarrow{\mathcal{F}}  \frac{1}{\left| a \right|} F(\frac{\omega}{a})  \\
    &\int \limits^{\infty}_{-\infty} f(at) e^{-i\omega t}dt = \\
    & = \left| u = at; \, dt = \frac{1}{a}du; \, \text{change of limits $\rightarrow$ absolute value of $a$ }\right| =\\
    & = \frac{1}{\left| a \right|} \int \limits^{\infty}_{-\infty} f(u) e^{-i \frac{\omega}{a} u }du =  \\
    & =\frac{1}{\left| a \right|} F(\frac{\omega}{a})
\end{align*}

\paragraph{Time shift} Shift in time - phase shift in frequency. Follows from the change of variables:
\begin{align*}
    &f(t - t_0) \xrightarrow{\mathcal{F}}  e^{-i\omega t_0} F({\omega})  \\
    &\int \limits^{\infty}_{-\infty} f(t - t_0) e^{-i\omega t}dt = \left| u=t - t_0 \right| = \\
    &=\int \limits^{\infty}_{-\infty} f(u) e^{-i\omega u - i \omega t_0}du = \\ 
    &=e^{- i \omega t_0}\int \limits^{\infty}_{-\infty} f(u) e^{-i\omega u }du =  e^{-i\omega t_0} F({\omega})
\end{align*}

\paragraph{Differentiation} Derivative in time - multiplication by $i \omega$ in frequency. 
Useful property for solving differential equations as it transform differential equation into algebraic.
 Follows from the inverse transform:
\begin{align*}
    & \frac{d f(t)}{dt} \xrightarrow{\mathcal{F}} i\omega F({\omega}) \\
    &\frac{d f(t)}{dt} = \frac{d }{dt} \int \limits^{\infty}_{-\infty} F(\omega) e^{i\omega t}d \omega = \\
    &=\int \limits^{\infty}_{-\infty}  F(\omega) \frac{d e^{i\omega t}}{dt} d \omega = \\
    &= \int \limits^{\infty}_{-\infty} \left[ i\omega F(\omega) \right] e^{i\omega t} d \omega \\
\end{align*}

\paragraph{Convolution} Convolution in time - multiplication in frequency. 
Useful in working with Linear Time Invariant systems, discretization, optics etc.
 Effect of the LTI systems is described as a convolution of the input signal with its impulse response.
 Follows from the changing the order of integration:
\begin{align*}
    % TODO proof
    f(t) * g(t) \xrightarrow{\mathcal{F}} \int \limits^{\infty}_{-\infty} \int \limits^{\infty}_{-\infty} f(\tau ) g(t - \tau) d \tau e^{-i\omega t}dt = 
\end{align*}

\paragraph{Symmetry} Follows from the definitions:
\begin{align*}
    &f(t) \xrightarrow{\mathcal{F}} F(\omega) \\
    &F(t) \xrightarrow{\mathcal{F}} f(\omega)
    % TODO proof
\end{align*}

\paragraph{Multiplication} Multiplication in time - convolution in frequency. Follows from the symmetry:
\begin{align*}
    f(t)g(t) \xrightarrow{\mathcal{F}} F(\omega) * G(\omega)
    % TODO proof
\end{align*}

\paragraph{Integration} Reverse of the differentiation. 
The term with Dirac delta function takes care of the constant of integration with a constant in time part of the function.
\begin{align*}
    \int \limits^{t}_{-\infty}f(\tau)d \tau \xrightarrow{\mathcal{F}} \frac{F(\omega)}{i \omega} + \frac{1}{2}F(0)\delta(t)
    % TODO proof
\end{align*}

\paragraph{Modulation} Multiplication with a sine wave in time - shift in frequency. Basis of AM modulation.
Follows from the time-shift and symmetry:
\begin{align*}
    e^{i \omega_0 t}f(t) \xrightarrow{\mathcal{F}} F(\omega - \omega_0)
    % TODO proof
\end{align*}

\paragraph{Conjugate signal}
\begin{align*}
   f^*(t) \xrightarrow{\mathcal{F}} F^{*}(-\omega)
    % TODO proof
\end{align*}

\subsection{Important functions}

\paragraph{Dirac delta function}

\paragraph{sin and cos}

\paragraph{rect}

\paragraph{sinc}

\paragraph{Grid of delta functions}


\subsection{Fourier series}
Functions $e^{i n \omega t}$ for $n \in \mathbb{Z}$ form an orthonormal basis in Hilbert space.
So Fourier transform is a basis expansion of a $f(t)$ function.
And in Fourier series coefficients are the inner products of a $f(t)$ with basis functions.
Fourier series expansion is:
\begin{align*}
    &f(t) = \sum \limits^{\infty}_{n=- \infty} c_n e^{i n \omega t}, \, \text{where} \\
    &c_n = \left< f(t), e^{i n \omega t} \right> = \int \limits^{\infty}_{-\infty} f(t) e^{-i n \omega t}dt
\end{align*}


\section{Nyquist-Shannon-Kotelnikov theorem}

Paraphrased from \cite{nyquist_certain_1928,shannon_communication_1949,kotelnikov_transmission_2006}:
\begin{displayquote}
Any function $f(t)$ consisting of frequencies from $0$ to $f_1$ (band-limited) can be 
reconstructed with arbitrary accuracy from discrete samples at intervals of no higher than $\frac{1}{2f_1}$.
\end{displayquote}

\section{Signal energy}
% TODO:
Factor of $\frac{1}{2 \pi}$ is required for the normalization and holding of Plancherel's theorem for energy relation.


\section{Definitions}
Complex numbers:
\begin{align*}
    &z = a + ib = r e^{i\phi} \\
    &z^* = a - ib \\
    &\left| z \right| ^2 = z z^* = a^2 + b^2
\end{align*}
Inner product of complex functions:
\begin{equation*}
    \left< f(t),g(t) \right> = \int \limits^{\infty}_{-\infty} f(t) g^{*}(t) dx
\end{equation*}
Convolution:
\begin{equation*}
f(t) * g(t) =  \int \limits^{\infty}_{-\infty} f(\tau ) g(t - \tau) d \tau
\end{equation*}
Dirac delta function:
\begin{equation*}
    \delta(t) =  \begin{cases}
        \infty, \, \text{if $t = 0$};\\
        0, \, \text{otherwise}
    \end{cases}
\end{equation*}
% TODO: Dirac function properties, Dirac grid etc.

\bibliographystyle{IEEEtran}
\bibliography{signal_processing}

\end{document}
