\documentclass{article}
\usepackage{graphicx} % Required for inserting images
\usepackage{amsmath} 
\usepackage{pdflscape}
\usepackage{tikz}
\usetikzlibrary{calc}
\usepgflibrary{arrows.meta}
\usetikzlibrary{tikzmark}

\title{Vectors and Functions}
\author{Airat Galiamov}
\date{November 2024}

\begin{document}

\maketitle

\section{Vector basis}

 \pgfmathsetmacro{\graphsize}{5}
 \pgfmathsetmacro{\xone}{-.4-\graphsize}
 \pgfmathsetmacro{\xtwo}{ \xone+2*\graphsize}
 \pgfmathsetmacro{\yone}{-.4}
 \pgfmathsetmacro{\ytwo}{\yone+\graphsize}

\begin{tikzpicture}[scale=1]
    % Draw grid
    % \draw[step=1,gray,very thin] (\xone,\yone) grid (\xtwo,\ytwo);
    % Draw axes
    \coordinate (i_vec) at (1,0);
    \coordinate (j_vec) at (0,1);
    \coordinate (e_1_vec) at (0.707,0.707);
    \coordinate (e_2_vec) at (-0.707,0.707);
    \coordinate (center) at (0,0);
    \coordinate (a_vec) at (3.7,2.8);
    \draw [<->, thick] ($ \graphsize *(i_vec) $) node (xaxis) [above] {$x$}
        -- (center) node [below left] {$O$}
        -- ($ \graphsize *(j_vec) $) node (yaxis) [right] {$y$};
    % Draw i,j basis vectors
    \coordinate (e_center) at (\graphsize,\graphsize);
    \draw [<->, very thick] ($ (i_vec) + (e_center) $) node (i_vect) [above right] {$\vec{i}$}
    -- (e_center)
    -- ($ (j_vec) + (e_center) $) node (j_vect) [above right] {$\vec{j}$};
    % Draw e_1, e_2 basis vectors
    \draw [<->, very thick, blue] ($ (e_1_vec) + (e_center) $) node (e_1_vect) [above] {$\vec{e}_1$}
    -- (e_center)
    -- ($ (e_2_vec) + (e_center) $) node (e_2_vect) [above] {$\vec{e}_2$}; 
    % Draw x' and y'
    \draw [<->, thick, blue] ($ \graphsize *(e_1_vec) $) node (xpaxis) [above] {$x^\prime$}
        -- (center)
        -- ($ \graphsize *(e_2_vec) $) node (ypaxis) [above] {$y^\prime$};
    \draw [-, blue] (center) -- ($ -1*(e_2_vec) $);
    % Vector a
    \draw [->, very thick]  (center) -- (a_vec) node (a) [right] {$\vec{a}$};
    \coordinate (a_1) at ($(center)!(a_vec)!(i_vec)$);
    \coordinate (a_2) at ($(center)!(a_vec)!(j_vec)$);
    \coordinate (a_p_1) at ($(center)!(a_vec)!(e_1_vec)$);
    \coordinate (a_p_2) at ($(center)!(a_vec)!(e_2_vec)$);
    \draw [dotted,-{Rectangle[]}] (a_vec) --  (a_1);
    \draw [dotted,-{Rectangle[]}] (a_vec) -- (a_2);
    \draw [dotted, blue,-{Rectangle[]}] (a_vec) -- (a_p_1);
    \draw [dotted, blue,-{Rectangle[]}] (a_vec) -- (a_p_2);
    \node at (a_1) [below] {$a_x$};
    \node at (a_2) [left] {$a_y$};
    \node at (a_p_1) [above left, blue] {$a_1$};
    \node at (a_p_2) [below left, blue] {$a_2$};
    
    \fill[black] (center) circle (2pt);
    \fill[black] (e_center) circle (1pt);

\end{tikzpicture}

Inner product: $ \vec{a} \cdot \vec{b} = \left< \vec{a}, \vec{b} \right> = \vec{a}^T \cdot \vec{b}$.

Orthonormal basis $\vec{i}, \vec{j}$:
\begin{align*}
\left< \vec{i}, \vec{j} \right> = 0 \\
\left|\vec{i}\right|^2 = 1 \\
\left|\vec{j}\right|^2 = 1
\end{align*}

Orthonormal basis $\vec{e}_1,  \vec{e}_2$:
\begin{align*}
&\vec{e}_1 = e_{x,1} \cdot \vec{i} + e_{y,1} \cdot \vec{j} =   \begin{bmatrix}  e_{x,1} \\ e_{y,1}\end{bmatrix} \\
&\vec{e}_2 = e_{x,2} \cdot \vec{i} + e_{y,2} \cdot \vec{j} =   \begin{bmatrix}  e_{x,2} \\ e_{y,2}\end{bmatrix} \\
\left< \vec{e}_1,  \vec{e}_2 \right>  &= \left[  e_{x,1} \,  e_{y,1} \right] \cdot  \begin{bmatrix}  e_{x,2} \\ e_{y,2}\end{bmatrix} = 
e_{x,1} \cdot e_{x,2} + e_{y,1} \cdot e_{y,2} = 0  \\
\left| \vec{e}_n \right| ^2 &= \left< \vec{e}_n,  \vec{e}_n \right> = \left[  e_{x,n} \,  e_{y,n} \right] \cdot  \begin{bmatrix}  e_{x,n} \\ e_{y,n}\end{bmatrix} = 
e_{x,n}^2 + e_{y,n}^2 = 1
\end{align*}
It follows from orthonormality:
\begin{align} \label{eq:orthonormal}
    \begin{cases}
        e_{x,1} = e_{y,2} \\
        e_{x,2} = -e_{y,1}
    \end{cases} \textnormal{or} \,
    \begin{cases}
        e_{x,1} = -e_{y,2} \\
        e_{x,2} = e_{y,1}
    \end{cases}
\end{align}


Example:
\begin{align*}
    \vec{e}_1 &= \frac{1}{\sqrt{2}}  \begin{bmatrix} 1 \\ 1\end{bmatrix} \\
    \vec{e}_2 &= \frac{1}{\sqrt{2}} \begin{bmatrix}  -1 \\ 1\end{bmatrix}
\end{align*}

Basis expansion - linear combination of basis vectors:
Example for a given vector $\vec{a}$:
\begin{align*}
    \vec{a}  = a_x \cdot \vec{i} + a_y \cdot \vec{j} = \begin{bmatrix}  a_x \\ a_y\end{bmatrix}
\end{align*}
Changing basis to $\vec{e}_n$:
\begin{align*}
    \vec{a}  &= a_1 \cdot \vec{e}_1 + a_2 \cdot \vec{e_2} = a_1 \cdot  \begin{bmatrix}  e_{x,1} \\ e_{y,1}\end{bmatrix}  + a_2 \cdot  \begin{bmatrix}  e_{x,2} \\ e_{y,2}\end{bmatrix} = \\
    &=  \begin{bmatrix}  a_1 \cdot e_{x,1} + a_2 \cdot e_{x,2} \\  a_1 \cdot e_{y,1} + a_2 \cdot e_{y,2}\end{bmatrix} = \begin{bmatrix}  a_x \\ a_y\end{bmatrix}\\
    a_x &= a_1 \cdot e_{x,1} + a_2 \cdot e_{x,2} \\
    a_y &= a_1 \cdot e_{y,1} + a_2 \cdot e_{y,2} 
\end{align*}
Solve to find $a_1$ and $a_2$. Remember in orthonormal basis:
\begin{align*}
    a_1 &= \left< \vec{a}, \vec{e}_1 \right> = \left[ a_x \, a_y \right] \cdot \begin{bmatrix}  e_{x,1} \\ e_{y,1}\end{bmatrix} = \\
     &= a_x \cdot e_{x,1} + a_y \cdot e_{y,1} \\
    a_2 &= \left< \vec{a}, \vec{e}_2 \right> = \left[ a_x \, a_y \right] \cdot \begin{bmatrix}  e_{x,2} \\ e_{y,2}\end{bmatrix} = \\
     &= a_x \cdot e_{x,2} + a_y \cdot e_{y,2}
\end{align*}
Check it is correct (remember \ref{eq:orthonormal}):
\begin{align*}
    a_x &= a_1 \cdot e_{x,1} + a_2 \cdot e_{x,2} = \\
    &= (a_x \cdot e_{x,1} + a_y \cdot e_{y,1})  \cdot e_{x,1} + (a_x \cdot e_{x,2} + a_y \cdot e_{y,2})   \cdot e_{x,2}  = \\
    &= a_x \cdot e_{x,1}^2 + a_y \cdot e_{y,1}  \cdot e_{x,1} + a_x \cdot e_{x,2}^2 + a_y \cdot e_{y,2}   \cdot e_{x,2} = \\
    &= a_x \cdot (e_{x,1}^2 + e_{x,2}^2)  + a_y \cdot  (e_{y,1}  \cdot e_{x,1} + e_{y,2}   \cdot e_{x,2}) = \\
    &= a_x \\
    a_y &= ... \\
\end{align*}

For example vector:
\begin{align*}
    &\vec{a} = \begin{bmatrix}
        37 \\ 28
    \end{bmatrix} \\
    &\vec{e}_1 = \frac{1}{\sqrt{2}}  \begin{bmatrix} 1 \\ 1\end{bmatrix}  \textnormal{and} \,
    \vec{e}_2 = \frac{1}{\sqrt{2}} \begin{bmatrix}  -1 \\ 1\end{bmatrix} \\
    &a_1 = a_x \cdot e_{x,1} + a_y \cdot e_{y,1} = 37  \frac{1}{\sqrt{2}} + 28  \frac{1}{\sqrt{2}} = 65 \frac{1}{\sqrt{2}} \\
    &a_2 = a_x \cdot e_{x,2} + a_y \cdot e_{y,2} = - 37  \frac{1}{\sqrt{2}} + 28  \frac{1}{\sqrt{2}} = -9 \frac{1}{\sqrt{2}} \\
    &\vec{a} = 65 \frac{1}{\sqrt{2}} \cdot \vec{e}_1 - 9 \frac{1}{\sqrt{2}} \vec{e}_2
\end{align*}

\begin{landscape}
{\renewcommand{\arraystretch}{2}
\begin{tabular}{ c | c | c }
  & Vectors & Complex Functions \\ 
  \hline
 Elements & $\vec{a} = \begin{bmatrix}
 a_x\\
 a_y\\
 \vdots\\
 a_z\end{bmatrix}$ & $\mathbf{F}(x)$ \\  
 Inner product & $\vec{a} \cdot \vec{b} = \left< \vec{a}, \vec{b} \right> = \vec{a}^T \cdot \vec{b} =  a_x \cdot b_x + \hdots + a_z \cdot b_z  = \sum\limits^{z}_{i=x} a_i \cdot b_i $ & $\left<\mathbf{F}|\mathbf{G}\right> = \int\limits^{\infty}_{-\infty} \mathbf{F} \cdot \mathbf{G}^{*} dx$   \\
 Linear & $(k\vec{a} + t\vec{b}) \cdot \vec{c} = k (\vec{a} \cdot \vec{c}) + t (\vec{b} \cdot \vec{c}) $  & $\left<k\mathbf{F} + t\mathbf{G}|\mathbf{H}\right> = k\left<\mathbf{F} |\mathbf{H}\right> + t\left<\mathbf{G} |\mathbf{H}\right> $ \\
 
  Orthogonal & $\vec{a}  \perp \vec{b}$ \, if $\vec{a} \cdot \vec{b} = 0 $ &  $\mathbf{F}  \perp \mathbf{G}$ \, if $\left<\mathbf{F}|\mathbf{G}\right> = 0 $ \\
 Norm$^2$ & $\left| \vec{a} \right| ^2 = \vec{a} \cdot \vec{a} $ &  $\left| \mathbf{F} \right| ^2 = \left<\mathbf{F}|\mathbf{F}\right>$ \\
 Normalized &  $\left| \vec{a} \right| ^2 = 1$ &$\left| \mathbf{F} \right| ^2 = 1$ \\

 Orthonormal basis  & $\vec{e}_n$; \, $\left|\vec{e}_n\right|^2 =1$; \, $\vec{e}_n \cdot \vec{e}_m = 0$, \, if $n \neq m$  & $\mathbf{\Phi}_n$; \, $\left|\mathbf{\Phi}_n\right|^2 =1$; \, $\left<\mathbf{\Phi}_n | \mathbf{\Phi}_m\right> =0$, \, if $n \neq m$ \\
 Basis expansion, function series & $\vec{a} = c_1 \vec{e}_1 + \hdots + c_n \vec{e}_n = \sum\limits^{n}_{i=1} c_i \vec{e}_i$ &
 $\mathbf{F}(x) = c_1 \mathbf{\Phi}_1 + \hdots + c_n \mathbf{\Phi}_n =  \sum\limits^{n}_{i=1} c_i \mathbf{\Phi}_i $
 \\
Series coefficients  & 
$c_i = \vec{a} \cdot \vec{e}_i $ & $c_i = \left<\mathbf{F} | \mathbf{\Phi}_i\right> = \int\limits^{\infty}_{-\infty} \mathbf{F} \cdot \mathbf{\Phi}_i^{*} dx  $
\\
&
$\begin{array}{rll}
\vec{a} \cdot \vec{e}_i  &= (c_1 \vec{e}_1 + \hdots + c_n \vec{e}_n) \cdot  \vec{e}_i = \\
&=  c_1 \vec{e}_1 \cdot \vec{e}_i + \hdots + c_i \vec{e}_i \cdot \vec{e}_i  + \hdots + c_n \vec{e}_n  \cdot \vec{e}_i = \\
&= c_i
\end{array}$
&
$
\begin{array}{rll}
\left<\mathbf{F} | \mathbf{\Phi}_i\right> &= \left<(c_1 \mathbf{\Phi}_1 + \hdots + c_n \mathbf{\Phi}_n) | \mathbf{\Phi}_i\right>  = \\
&= c_1\left<\mathbf{\Phi}_1 | \mathbf{\Phi}_i\right>  + \hdots +c_i\left<\mathbf{\Phi}_i | \mathbf{\Phi}_i\right>  + \hdots c_n\left<\mathbf{\Phi}_n | \mathbf{\Phi}_i\right> = \\
&= c_i
\end{array}
$
\\

\end{tabular}
}
\end{landscape}


\end{document}
